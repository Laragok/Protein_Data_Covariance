\documentclass[14pt]{beamer}

% Presento style file
\usepackage{config/presento}
\usepackage{multicol}
% custom command and packages
\input{config/custom-command}

% Information
\title{Protein Covariance:}
\subtitle{Predicting Phenotypes Based On Amino Acid Sequences}
\author{Lara Gokcelioglu}
\institute{Institute For Computing In Research Santa Fe}
\date{\today}

\begin{document}

% Title page
\begin{frame}[plain]
\maketitle
\end{frame}

% sections in the presentation
\begin{frame}{Project Overview}
 \color{colorblue}\begin{itemize}
  \item\begin{center}Mix of \color{colororange}\textbf{biology and computing}\end{center}
  \item\begin{center}Phenotypes of proteins \color{colororange}\textbf{based on their amino acid sequences}.\end{center}
  \item\begin{center}6 different models \color{colororange}\textbf{for 3 different phenotypes and 2 representations each}.\end{center}
 \end{itemize}
\end{frame}

\begin{frame}{Dataset}
 \begin{fullpageitemize}
      \item\color{colorgreen}\begin{center}\largetext{The dataset consists of \color{colororange}\textbf{227 proteins }\color{colorgreen} and each protein’s \color{colororange}\textbf{amino acid sequence}\color{colorgreen}--which later gets \color{colororange}\textbf{fixed to 512 positions.}}\end{center}
 \end{fullpageitemize}
\end{frame}

\begin{frame}
\begin{figure}
        \centering
        \includegraphics[width=1.0\linewidth]{dataset.png}
        \caption{Partial Dataset}
        \label{fig:enter-label}
    \end{figure}    
\end{frame}

\begin{frame}{Phenotypes}
 \begin{fullpageitemize}
  \item\color{colorblue}\begin{center}\largetext{The first phenotype is \color{colororange}\textbf{em\_max}}\end{center}
  \item \begin{center}\largetext{The second phenotype is \color{colororange}\textbf{ex\_max}}\end{center}
  \item \begin{center}\largetext{The third and last phenotype is \color{colororange}\textbf{states\_0\_brightness}}\end{center}
 \end{fullpageitemize}
\end{frame}

\begin{frame}{Representations}
 \begin{fullpageitemize}
  \item\begin{center}\largetext{\color{colorgreen}\textbf{There are two representations of the data:}} \end{center}
  \item\color{colororange} \begin{center}\largetext{pc\_coords}\end{center}
  \item\color{colororange} \begin{center}\largetext{proteins\_projected\_pc\_coords}\end{center}
 \end{fullpageitemize}
\end{frame}

\begin{frame}{\textbf{Data Manipulation}}
 \begin{fullpageitemize}
  \item\color{colorblue}\begin{center}\largetext{\textbf{aminoacids}}\end{center}
  \item\color{colorblue}\begin{center}\largetext{\textbf{leftjustified\_seqs}}\end{center}
  \item\color{colorblue}\begin{center}\largetext{\textbf{match\_aminoacids}}\end{center}
  \item\color{colorblue}\begin{center}\largetext{\textbf{Singular Value Decomposition}}\end{center} 
 \end{fullpageitemize}
\end{frame}

\begin{frame}{Aminoacids}
    \begin{figure}
        \centering
        \includegraphics[width=1.0\linewidth]{aminoacids.png}
        \caption{aminoacids Variable}
        \label{fig:enter-label}
    \end{figure}
\end{frame}

\begin{frame}{leftjustified\_seqs}
    \begin{figure}
        \centering
        \includegraphics[width=1.0\linewidth]{leftjustified.png}
        \caption{leftjustified\_seqs variable}
        \label{fig:enter-label}
    \end{figure}
\end{frame}

\begin{frame}{match\_amoinoacids Function}
    \begin{figure}
        \centering
        \includegraphics[width=1.0\linewidth]{match_aminoacids.png}
        \caption{match\_aminoacids function}
        \label{fig:enter-label}
    \end{figure}
   \end{frame}

\begin{frame}{Singular Value Decomposition}
    \begin{figure}
        \centering
        \includegraphics[width=1.0\linewidth]{svd.png}
        \caption{Application Of SVD}
        \label{fig:enter-label}
    \end{figure}
\end{frame}



\begin{frame}{Training The Model}
    \begin{itemize}
    \item
    \item\color{colorblue}\largetext\textbf{K-Fold Validation}
        \begin{itemize}
         \item
         \item\color{colorgreen}\largetext\textbf{splits k=10}
         \item
         \item\begin{center}\largetext\textbf{90\% train set vs. 10\% test set}\end{center} 
         \item
         \end{itemize}
 \end{itemize}
 \end{frame}

\begin{frame}
       \begin{figure}
            \centering
            \includegraphics[width=0.7\linewidth]{training1.png}            \label{fig:enter-label}
           \section{training 2}
            \begin{figure}
                \centering
                \includegraphics[width=0.7\linewidth]{training2.png}
                \label{fig:enter-label}
            \end{figure}
        \end{figure}
\end{frame}

\begin{frame}{K-Fold Cross Validation}
    \begin{fullpageitemize}
  \item\color{colorgreen}\begin{center}\largetext{Increased \color{colororange}\textbf{accuracy}}\end{center}
  \item\color{colorgreen}\begin{center}\largetext{\color{colororange}\textbf{More data }\color{colorgreen}for both sets}\end{center}
  \item\color{colorgreen}\begin{center}\largetext{\color{colororange}\textbf{Reduces variance}}\end{center}
    \end{fullpageitemize}
\end{frame}

\begin{frame}{$R^2$ Value}
    \begin{itemize}
        \item\color{colorblue}\largetext{The strength of the \color{colororange}\textbf{relationship between the model and the dependent variable}}
        \begin{figure}
            \centering
            \includegraphics[width=0.5\linewidth]{r2.png}
            \label{fig:enter-label}
        \end{figure}
        \item\color{colorblue}\largetext{Coefficient of determination}

    \end{itemize}
\end{frame}

\begin{frame}{Analysis Of Em\_Max}
    \begin{figure}
     \centering
     \includegraphics[width=0.75\linewidth]{Screenshot at 2024-07-28 11-33-26.png}
     \caption{Analysis Of Most Efficient Emission Wavelength}
     \label{fig:enter-label}
    \end{figure}
\end{frame}

\begin{frame}{Analysis Of Projected Em\_Max}
    \begin{figure}
        \begin{figure}
       \centering
       \includegraphics[width=0.75\linewidth]{projectedemmax.png}
       \caption{Analysis Of Most Efficient Emission Wavelength In Projected Representation}
       \label{fig:enter-label}
        \end{figure}
    \end{figure}
\end{frame}

\begin{frame}{Analysis Of Ex\_Max}
    \begin{figure}
     \centering
     \includegraphics[width=0.75\linewidth]{exmax.png}
     \caption{Analysis Of Most Efficient Excitation Wavelength}
     \label{fig:enter-label}
    \end{figure}
    \end{frame}

\begin{frame}{Analysis Of Projected Ex\_Max}
    \begin{figure}
     \centering
     \includegraphics[width=0.50\linewidth]{projectedexmax.png}
     \caption{Analysis Of Most Efficient Excitation Wavelength In Projected Representation}
     \label{fig:enter-label}
    \end{figure}
\end{frame}

\begin{frame}{Analysis Of States\_0\_Brightness}
    \begin{figure}
     \centering
     \includegraphics[width=0.50\linewidth]{brightness.png}
     \caption{Analysis of Brightness}
     \label{fig:enter-label}
    \end{figure}
\end{frame}

\begin{frame}{Analysis Of Projected States\_0\_Brightness}
    \begin{figure}
     \centering
     \includegraphics[width=0.50\linewidth]{projectedbrightness.png}
     \caption{Analysis of Brightness In Projected Representation}
     \label{fig:enter-label}
    \end{figure}
\end{frame}

\begin{frame}{pc\_coords $R^2$ Values}
    \begin{figure}
        \centering
        \includegraphics[width=0.75\linewidth]{Figure_1.png}
        \label{fig:enter-label}
    \end{figure}
\end{frame}

\begin{frame}{proteins\_projected\_pc\_coords $R^2$ Values}
    \begin{figure}
        \centering
        \includegraphics[width=0.75\linewidth]{Figure_3.png}
        \label{fig:enter-label}
    \end{figure}
\end{frame}

\begin{frame}{Future Path}
    \begin{fullpageitemize}
        \item\color{colorgreen} \begin{center}\largetext{Considering more folds in analysis}\end{center}
        \item\color{colorgreen} \begin{center}\largetext{Moving beyond linear models}\end{center}
        \item\color{colorgreen} \begin{center}\largetext{Including more parameters}\end{center}
    \end{fullpageitemize}
\end{frame}

\begin{frame}{Challenges}
    \begin{fullpageitemize}
        \item\color{colorblue} \begin{center}\largetext{Choosing the best method to train a model on}\end{center}
        \item\color{colorblue} \begin{center}\largetext{Lack of answers}\end{center}
        \item\color{colorblue} \begin{center}\largetext{Inaccurate results}\end{center}
    \end{fullpageitemize}
\end{frame}

\begin{frame}{References}
    \begin{itemize}
        \item\color{colorblue}(N.d.). Statisticsbyjim.com. Retrieved July 28, 2024, from https://statisticsbyjim.com/regression
        /interpret-r-squared-regression/
        \item
    \end{itemize}
\end{frame}

\end{document}
